\section{Лекция 7 05.05.25}
Линейное уравнение Вольтерры II рода:
\[y(x) -  \int_{a}^{x} K(x,s)y(s) \,ds  = f(x),\space x\in[a,b] \] 
Нелинейное уравнение Вольтерры II рода:
\[y(x) -  \int_{a}^{x} K(x,s,y(s)) \,ds  = f(x),\space x\in[a,b] \] 
\(K(x,s)\) --- ядро интегрального уравнения; уравнение интегральное, т.к. \(y(x)\) стоит под знаком интеграла.\\
Линейное уравнение Фредгольма II рода: 
\[y(x) -  \lambda\int_{a}^{b} K(x,s)y(s) \,dx  = f(x),\space x\in[a,b] \] 
// Уравнение Вольтерры I рода: \( f(t) = \mathlarger{\int}_{a}^{x} K(x,s)y(s)\,ds\)\\// Вольтерры с точки зрения математики --- частный случай Фредгольма.\\\\
\underline{Откуда возникают эти уравнения}\\
К сожалению, у нас плохие знания физики; например, уравнение Фредгольма II рода описывает колебательные движения нити, на которую действуют некие силы, а более простое интегральное уравнение --- её прогиб.
\begin{itemize}
    \item \textbf{Простой пример:} И.Г.Петровский <<Лекции по теории интегральных уравнений>>
    \item Очень большой обзор: С.Г. Михлин <<Приложения интегральных уравнений к проблемам механики, матфизики и техники>> (1947г)
\end{itemize}
Михлин --- математик, но тогда еще вычмат неразрывно связан с физикой, а сейчас про приложения в книгах по вычам не пишут.
\subsection{Уравнения Вольтерры}
Известно, что:
\begin{enumerate}
    \item \( f \equiv 0\) (однородное уравнение) --- только тривиальное решение
    \item Если \(f(x) \in C[a,b]\) и \(K(x,s)\) непр. в $\triangle \, \begin{cases} 
    x=b\\
    x=s\\
    s=a
    \end{cases} \Rightarrow \exists!$ решение
\end{enumerate}
\textbf{!} Вообще говоря, рассматривая все последующие численные методы, мы исходим из предположения $\exists$ и $!$ решения.
\[y(x) -  \int_{a}^{x} K(x,s)y(s) \,ds  = f(x) \]
\subsubsection{Метод квадратур}
Замена интеграла квадратурной суммой: 
\[\int_{a}^{b} g(x) \,dx = \sum_{i=0}^{n}w_ig(x_i) + R; \, a\le x_0\le x_1 \le ...\le x_n\le b\]
где $w_i$ --- веса, а $R$ --- ошибка.\\Подставим для \(i=0,1,..,n\) \space \(x=x_i\):
\[y(x_i) -  \int_{a}^{x_i} K(x_i,s)y(s) \,ds  = f(x_i) \]
Этот интеграл уже можно заменить квадратурной суммой:
\[y(x_i)-\sum_{j=0}^{i}w_jK(x_i,x_j)y(x_j)=f(x_i) + R_i\]
(сумма до $i$, a не до $n$, т.к. интеграл до $x$)\\
Пусть \( K(x_i,x_j)=K_{ij}, f(x_i)=f_i, y(x_i)=y_i\); и отбросим $R_i$, считая их малыми:
\[y_i-\sum_{j=0}^{i}w_jK_{ij}y_j=f_i;\, i=0,1,..,n\]
Перепишем в виде
\[ \sum_j^{i-1}w_jK_{ij}y_j+(1-w_iK_{ii})y_i=f_i \Rightarrow\]
\[ \Rightarrow \begin{pmatrix}
1-w_0K_{00}  \\
-w_0K_{10}&1-w_1K_{11}& &\mbox{\Huge 0}\\
\vdots & \vdots & \ddots &\\
-w_0K_{n0}&-w_1K_{n1}&\cdots&1-w_nK_{nn}\\
\end{pmatrix} \begin{pmatrix}
y_0\\
y_1\\
\vdots\\
y_n
\end{pmatrix} = \begin{pmatrix}
f_0\\
f_1\\
\vdots\\
f_n
\end{pmatrix} \]
Решение: 
\[y_i=\frac{f_i+\mathlarger{\sum}_{j=0}^{i-1}w_jK_{ij}y_j}{1-w_iK_{ii}}	\rightsquigarrow \textrm{ последовательно находим по этой формуле}\]
\(1-w_iK_{ii}\ne 0 \iff w_i \ne \frac{1}{K_{ii}}\) --- добиться выбором узлов кв. ф.\\\\
\underline{Упражнение:} на равномерной сетке с шагом $h$ при использовании метода трапеций формула имеет вид:
\[y_i=\frac{f_i+\frac{h}{2}K_{i0}{y_0} +h\mathlarger{\sum}_{j=1}^{i-1}K_{ij}y_j}{1-\frac{h}{2}K_{ii}}\]
Ну а после того, как вычислили $y(x_i)$, можно применить интерполяцию и вычислить в остальных точках.\\\\
\underline{Замечание:} \begin{itemize}
    \item численное интегрирование применили
    \item СЛАУ применили
    \item интерполяцию применили
\end{itemize}
Сами по себе эти задачи не всегда ценны, но как часть целого --- более чем.

\subsubsection{Метод квадратур для нелинейного уравнения}
\[y(x) -  \int_{a}^{x} K\mathlarger{(}x,s,y(s)\mathlarger{)} \,ds  = f(x),\space x\in[a,b] \] 
Исходим из предположения $\exists$ и $!$ решения
\[y(x_i) -  \int_{a}^{x_i} K\mathlarger{(}x,s,y(s)\mathlarger{)} \,ds  = f(x_i)\]
Заменяем конечными суммами:
\[y_i - \sum_{j=0}^{i}w_jK_{ij}(y_j) = f_i \]
\[y_i - w_iK_{ii}(y_i) = f_i + \sum_{j=0}^{i-1}w_jK_{ij}(y_j) \]
$	\rightsquigarrow$ последовательное решение нелинейных уравнений (а их мы на самой первой паре решали!)

\subsubsection{Метод простой итерации}
\[y(x) = f(x) +  \int_{a}^{x} K(x,s)y(s) \,ds ,\space x\in[a,b] \] 
\[y_k(x) = f(x) +  \int_{a}^{x} K(x,s)y_{k-1}(s) \,ds \textrm{ --- схема итераций} \]
Известно, что \begin{enumerate}
    \item при \(f(x) \in C[a,b]\) и \( K(x,s) \in C(\triangle \,\,\, a\le s\le x \le b)\) поточечно сходится к единственному решению при любом начальном приближении $y_0(x)$
    \item часто используют \(y_0(x) = f(x)\)
    \item скорость сходимости зависит от $K(x,s)$ и $f(x)$
\end{enumerate}
\underline{Численно:}\space
\(y_k(x_i) = f(x_i) +  \mathlarger{\int}_{a}^{x_i} K(x_i,s)y_{k-1}(s) \,ds \) $\rightarrow$ квадратурные суммы (на равномерной сетке)
\subsection{Линейное уравнение Фредгольма II рода}
\[y(x) -  \lambda\int_{a}^{b} K(x,s)y(s) \,ds  = f(x),\space x\in[a,b] \] 
Известно, что если \(\begin{cases}
    1)f\in C[a,b]\\
    2)K(x,s)\in C([a,b] \times[a,b])\\
    3)\textrm{при } f \equiv 0 \textrm{ имеется только тривиальное решение}(\Rightarrow \lambda \textrm{ регулярное}) \\
\end{cases}\\\Rightarrow \forall f(x)\) у интегрального уравнения Фредгольма II рода $\exists!$ непрерывное решение\\\\
\underline{Частные случаи:} \begin{enumerate}
    \item $K(x,s) = K(x-s)$ --- разностное ядро
    \item \(K(x,s) =\mathlarger{\sum}_{i=1}^{m}\alpha_{i}(x)\beta_{i}(s)\)--- вырожденное ядро ($\{\alpha_i\}, \{\beta_{i}\}$ --- ЛНЗ семейства)
\end{enumerate}
В таких случаях есть особые методы решения.
\subsubsection{Метод решения уравнения с вырожденным ядром}
\[y(x) -  \lambda\int_{a}^{b} \mathlarger{[}\sum_{i=1}^{m}\alpha_{i}(x)\beta_{i}(s)\mathlarger{]}y(s) \,ds  = f(x),\space x\in[a,b] \] 
Вытащим то, что не зависит от $s$ за знак интеграла:
\begin{align}
y(x) -  \lambda\sum_{i=1}^{m}\alpha_{i}(x)\int_{a}^{b} \beta_{i}(s)y(s) \,ds  = f(x) \tag{$\star$} \label{eq:star}
\end{align}
Пусть \(c_i=\mathlarger{\int}_a^b \beta_i(s)y(s)\,ds\) --- мы как-то научились их вычислять, тогда\\ \(y(x)=f(x) +\lambda \mathlarger{\sum}_{i=1}^mc_i\alpha_i(x),\space x\in[a,b]\). \underline{Никакой интерполяции.}\\
Подставим $y(x)$ в уравнение \eqref{eq:star}:
\[\sum_{i=1}^m\alpha_i(x)\mathlarger{\{} c_i-\int_a^b \beta_i(s)\mathlarger{[} f(s) + \lambda \sum_{j=1}^m c_j\alpha_j(s)\mathlarger{]}\,ds \mathlarger{\}} =0\]
т.к. $\{\alpha_i(x)\}_{i=1}^m$ --- ЛНЗ, все коэффициенты равны 0:
\[c_i-\int_a^b\beta_i(s)\mathlarger{[}f(s)+\lambda\sum_{j=1}^mc_j\alpha_j(s)\mathlarger{]}\, ds=0;\, i=1,..,m \]
\[c_i-\lambda\sum_{j=1}^m c_j\int_a^b \alpha_j(s)\beta_i(s)\,ds=\int_a^b\beta_i(s)f(s)\,ds\]
\[ a_{ij}=\int_a^b\alpha_j(s)\beta_i(s)\,ds;\, f_i=\int_a^b\beta_i(s)f(s)\,ds\]
\[c_i-\lambda\sum_{j=1}^ma_{ij}c_j=f_i\]
\[
\begin{pmatrix}
    1-\lambda a_{11} & -\lambda a_{12} & \cdots & -\lambda a_{1n}\\
    -\lambda a_{21} & 1-\lambda a_{22} & \cdots & -\lambda a_{2n}\\
    \cdots & \cdots & \cdots & \cdots\\
    -\lambda a_{m1} & -\lambda a_{m2} & \cdots & 1-\lambda a_{mm}
\end{pmatrix}
\begin{pmatrix}
    c_1\\c_2\\ \vdots \\ c_n
\end{pmatrix}
=
\begin{pmatrix}
    f_1\\ f_2 \\ \vdots \\f_n
\end{pmatrix}
\]
Известно, что если $\lambda$ --- регулярное, то система разрешима.\\
\underline{Итог:} \begin{itemize}
    \item вычисляем интегралы $a_{ij}$
    \item решаем СЛАУ
    \item получаем ответ \(y(x)=f(x)+\lambda \mathlarger{\sum}_{i=1}^m c_i\alpha_i(x)\)
\end{itemize}
\underline{Замечание:}\\
Если ядро невырожденное, но достаточно гладкое, его можно приблизить вырожденным.\\
\underline{Пример:}
\[y(x)+\int_0^1x(e^{xs}-1)y(s)\,ds=e^x-x\]
Известно, что есть решение y(x) \equiv 1.\\
\(K(x,s)=x(e^{xs}-1)\) --- невырожденное ядро. Но, разложив $e^{xs}$ в ряд Тейлора, получим вырожденное ядро.
\subsubsection{Метод сплайн-коллокаций}
Изучить самостоятельно дома и сделать задачку.
\subsection{Уравнение Фредгольма I рода}
А теперь рассмотрим уравнение Фредгольма I рода в условиях $\exists!$ решения:
\[ \int_a^bK(x,s)z(s)\,ds=u(x),\, \begin{cases}
    c\le x\le d\\
    z(s)\in C[a,b]\\
    u(x)\in L_2[c,d)\\
    \frac{\partial K}{\partial x}\in C \left ( \left \{ 
        \begin{align*}
  a\le x\le b \\ 
  c \le x \le d
\end{align*}
    \right \} \right ) 
\end{cases}\]
Попробуем применить предложенные методы (квадратур, сплайн-коллокаций) и ничего не выйдет...
\begin{itemize}
    \item почему?
    \item что делать?
\end{itemize}\\
\underline{Похожие примеры:}\\\\
\textcircled{\small{1}}\\
\[ \begin{cases}
        z_1+7z_2=5\\
        \sqrt{2}z_1+\sqrt{98}z_2 = \sqrt{50}
    \end{cases} \rightarrow\ \begin{cases}
        z_1 = 0,1\\
        z_2 = 0,7
    \end{cases} \]
Оставив два знака после запятой у коэффициентов при $z_1$ и $z_2$, получим:
\[ \begin{cases}
        z_1+7z_2=5\\
        1,41z_1+9,9z_2 = \sqrt{50}
    \end{cases} \rightarrow\ \begin{cases}
        z_1 = \frac{1}{3}\\
        z_2 = \frac{2}{3}
    \end{cases} \]
\[ \textrm{50 знаков:}\begin{cases}
        z_1 = 9,666..\\
        z_2 = 0.666..
    \end{cases} 
    \space
    \textrm{200 знаков:}\begin{cases}
        z_1 = -5,5\\
        z_2 = 1,5
    \end{cases} \]
Процесс никуда не сходится; малые возмущения исходных значений приводят к сильным возмущениям системы.\\
\textcircled{\small{2}} (c ним столкнулся Адамар)
\[\begin{cases}
    \frac{\partial^2u}{\partial x^2} + \frac{\partial^2u}{\partial y^2}=0\\
    u(x,0)=f(x)\\
    $\left.{\frac{\partial u}{\partial y}}\right|_{y=0}$ = \phi(x)\\
\end{cases}, \textrm{где } u, f, \phi \textrm{ непрерывны на своих областях определения}\]\\
$	\rightsquigarrow$ задача Коши для уравнения Лапласа\\\\
\(1^o) \begin{cases}
    f_1(x) \equiv 0\\
    \phi_1(x)\ = \frac{1}{a}sin(ax)
\end{cases} \Rightarrow u_1(x,y)=\frac{1}{a^2}sin(ax)sh(ay)\)\\
\(2^o) \begin{cases}
    f_2(x) \equiv 0\\
    \phi_2(x) \equiv0
\end{cases} \Rightarrow u_2(x,y)\equiv0\)
\[\rho_C(f_1,f_2)=\sup_{x}|f_1(x)-f_2(x)|=0\]
\[\rho_C(\phi_1,\phi_2)=\sup_{x}|\phi_1(x)-\phi_2(x)|=\frac{1}{a} \xrightarrow{a\rightarrow\infty}{0}\]
\[\rho_C(u_1,u_2)=\sup_{x,y}\left |\frac{1}{a^2}sin(ax)sh(ay)\right |= \sup_y\frac{1}{a^2}sh(ay)\xrightarrow{a\rightarrow\infty}{+\infty}\]
Снова малые возмущения данных краевой задачи сильно исказили решение.\\
Кажется, что мы готовы только к тем задачам, где нет <<искажений>>.\\\\
\underline{Определение:}\\
\(Az=u, \textrm{ где } z\in Z,\space u\in U,\space  A:Z\rightarrow U \) --- непрерывный оператор,\\
задача корректно поставлена (по Адамару), если \begin{enumerate}
    \item уравнение разрешимо $\forall u\in U$
    \item решение единственно
    \item решение устойчиво к возмущениям, т.е. малым возмущениям u в метрике U соответствуют малые возмущения z в метрике Z
\end{enumerate}\\
\underline{Что у уравнений Фредгольма I рода?}\\
Пусть при  \(u=u_0(x)\) имеем решение \( z=z_0(x)\) и пусть \(u_{N,w}(x)=u_0(x)+N\mathlarger{\int}_a^bK(x,s)sin\,ws\,ds\).
Рассмотрим \(z_{N,w}(s)=z_0(s)+Nsin\,ws\,\):
\[\rho_U(u_{N,w}, u_0)=|N| \left (\mathlarger{\int}_c^d \left [ \int_a^b K(x,s)sin\,ws\,ds \right ]^2\,dx \right )^\frac{1}{2} \xrightarrow{w\rightarrow \infty}{0}\]
\[\int_a^b K(x,s)sin\,ws\,ds \xrightarrow{w\rightarrow \infty}{0} \textrm{ (теория рядов Фурье)}\]
\[\rho_Z(z_{N,w},z_0) = \max_{S\in[a,b]} |Nsin\,ws|=N \sNeg[1mu]{\xrightarrow{w\rightarrow\infty}}0  \]
\(\Rightarrow\) задача поставлена некорректно (хотя бы потому, что проблемы с устойчивостью).\\\\
Некорректно поставленные задачи делятся на \begin{enumerate}
    \item регулярные (по Тихонову)
    \item нерегулярные
\end{enumerate}\\
\underline{Идея регуляризации}\\
Решить операторное уравнение \(Az=u\rightsquigarrow ||Au-z||_2^2+\alpha||u||_2\rightarrow \min\)\\
(минимизация первого слагаемого $\rightarrow$ минимизация невязки, неустойчивость;\\ минимизация второго слагаемого $\rightarrow$ повышение устойчивости, увеличение невязки;\\$\alpha$ --- коэффициент регуляризации (используется для балансировки))\\
$\rightsquigarrow$ решить новое операторное уравнение \((A^TA+\alpha L)z_{\alpha}=A^Tu\),
где $L$ --- симметричный и положительно определенный оператор.\\
Оказывается, что тут будет устойчивость.\\
Выбор $\alpha$: тоже целая теория.\\
А.Н. Тихонов, А.С.Леонов, А.Г. Ягола <<Нелинейные некорректные задачи>> (1995г.)\\\\
A.M. Wazwaz "The regularisation method for Fredholm Integral Equation of the I kind"  (2011г.):\\
ур-е I рода \(\mathlarger{\int}_a^bK(x,t)u(t)\,dt \rightarrow \alpha u_{\alpha}(x)+\mathlarger{\int}_a^bK(x,t)u_{\alpha}(t)\,dt = f(x)\) --- ур-е II рода, и решать стандартными методами.
\begin{itemize}
    \item ничего не доказал
    \item привел числ. примеры
    \item похоже, работает
    \item я использовал
    \item Бурова ставит под сомнения
\end{itemize}\\
Бывают интегральные уравнения Абеля, Урысона, бывают интегро-дифференциальные уравнения. Мы скипаем.\\
А.В. Лебедева, В.М. Рябов <<О регулязации решения интегральных уравнений первого рода с помощью квадратурных формул>>
