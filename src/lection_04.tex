\section{Лекция 4 31.03.25}

\subsection{Численное дифференцирование (продолжение)}

Имеем следующую формулу:
\begin{equation}
 f'(x_i) = \frac{f(x_{i+1}) - f(x_{i})}{h}
\end{equation}
Нас интересуют следующие вопросы:\\
1) Что такое h?\\
2) Как её выбирать?

\subsubsection{Полная погрешность}
Полная погрешность состоит из:\\
1) Погрешность численного метода.\\
2) Погрешность вычислений: $f_{i}\to \tilde{f}_i $ \\

Пусть $\delta: |\tilde{f}_i-f_i| \leq \delta $ \\
К тому же: $|f_{i+1}-\tilde{f}_{i+1}| \leq \delta $ \\

$E(\delta,h)$ - полная погрешность $:= |\tilde{f}(x_{i})-f'(x_{i})|=|\frac{\tilde{f}_{i+1} - \tilde{f}_{i}}{h} -f'(x_{i+1})| \leq |\frac{\tilde{f}_{i+1} - \tilde{f}_{i}}{h} - \frac{f_{i+1} - f_{i}}{h}| + |\frac{f_{i+1} - f_{i}}{h} -f'(x_{i})|  \leq  |\frac{\tilde{f}_{i+1} - f_{i+1}}{h}| + |\frac{\tilde{f}_{i} - f_{i}}{h}| + |\frac{f_{i+1} - f_{i}}{h} -f'(x_{i})|  \leq \frac{2\delta}{h} + \frac{M_2}{2}h $ \\
Погрешность $M_2$ численного метода можно получить аналогичным методом из прошлой лекции.


Какая тут проблема?\\
$\frac{\partial}{\partial h}(\frac{2\delta}{h} + \frac{M_2}{2}h) = \frac{-2\delta}{h^2} + \frac{M_2}{2}  \nrightarrow 0$ при $h \to 0$\\
Получается, что нет смысла устремлять $h$ к нулю.\\
Оптимальным значением $h$ будет $\frac{2\sqrt{\delta}}{\sqrt{M_2}}$\\


\textbf{Пример}\\
Для \texttt{double}: $\delta \approx 10^{-16}$, тогда оптимальная $h = 10^{-8}$

\subsection{Численное интегрирование}
Как считать определённый интеграл? Обычно мы пользуемся формулой Ньютона-Лейбница:
\begin{equation}
\int\limits_{a}^{b} f(x) \, dx = F(b) - F(a)
\end{equation}
Однако не всегда существует F.\\
Идея:\\
\[
f(x) \approx \sum\limits_{i=0}^{n}f(x_k)\gamma_k(x) \Rightarrow \int\limits_{a}^{b} f(x) \, dx  =  \sum\limits_{i=0}^{n} ( \underbrace{\int\limits_{a}^{b} \gamma_k(x) \, dx}_{c_k} ) f(x_k)\\
\]\\

\begin{equation}
\int_{a}^{b} f(x) \, dx  \equiv  \sum\limits_{i=0}^{n} {c_k}  f(x_k)
\end{equation}
Здесь $\gamma_k$ --- интегрируемые функции\\
$c_k$ --- весовые коэффициенты\\
(21) --- квадратурные формулы\\

\textbf{Определение}\\
Алгебраическая степени точности (АСТ) квадратурной формулы --- наибольшая степень полинома, такого, что для него формула точна. То есть:\\
\[\int\limits_{a}^{b} f(x) \, dx  \equiv  \sum\limits_{i=0}^{n} {c_k}  f(x_k)\]



\subsubsection{Как искать квадратуры:}
Если $\gamma_k(x)$ --- полиномы, то говорят о квадратурных формулах Ньютона-Котеса.\\
Рассмотрим квадратурные формулы при разном числе узлов.\\
\\
\textbf{n = 0:}\\
\[
\int\limits_{a}^{b} f(x) \, dx  \approx f(x_0)(b-a)
\]
Необходимо выбрать $x_0$. Интуитивно понятно, что подойдет $\frac{b+a}{2}$\\
Получаем \textit{формулу средних прямоугольников}.\\
\[
\int\limits_{a}^{b} f(x) \, dx  \approx f(\frac{a+b}{2})(b-a)
\]
Теперь оценим точность.\\
\[
f(x) = f(\frac{a+b}{2}) +f'(\frac{a+b}{2})(x- \frac{a+b}{2}) + f''(\xi)(x- \frac{a+b}{2})^2 , \xi \in [a;b]
\]
\[
R(f) = |\int\limits_{a}^{b} f(x) \, dx  - (b-a)f(\frac{a+b}{2})| = |\int\limits_{a}^{b} (f(x)-f(\frac{a+b}{2})) \, dx| =
\]
\[ = |\int\limits_{a}^{b} f'(\frac{a+b}{2})(x-\frac{a+b}{2}) \, dx + \int\limits_{a}^{b} f''(\xi)(x- \frac{a+b}{2})^2 \, dx| \leq |f''(\xi)|\int\limits_{a}^{b}(x- \frac{a+b}{2})^2 \, dx = \frac{M_2}{24}(b-a)^3
\]\\
\textbf{Замечание}\\
Оценка достижима при $g(x) = (x-\frac{a+b}{2})^2$\\
\\
\textbf{Замечание}\\
$\forall f(x) = px + q$ формула средних прямоугольников точна. Значит АСТ = 1.\\
\\
\\
\textbf{n = 1:}\\
\\
\[P_1(x) = \frac{x-b}{a-b}f(a) + \frac{x-a}{b-a}f(b)\]\\
\[
\int\limits_{a}^{b} P_1(x) \, dx = \frac{f(a)}{a-b}\int\limits_{a}^{b}(x-b)dx + \frac{f(b)}{b-a}\int\limits_{a}^{b}(x-a)dx = \frac{b-a}{2}(f(a) + f(b))
\]\\
Получаем \textit{формулу трапеций}.\\
\[
\int\limits_{a}^{b} f(x) \, dx = \frac{f(a)+f(b)}{2}(b-a)
\]\\
Аналогично можно показать достижимость оценки $|R(x)| \leq \frac{M_2}{2}(b-a)^3$ при $g(x) = (x-a)^2$.\\
\textbf{Замечание}\\
АСТ = 1, несмотря на то, что узлов больше.\\
\textbf{n = 2:}\\
$x_0 = a; x_1 = \frac{a+b}{2}; x_2 = b$\\
$[a;b] \to [0;\omega]$\\
$P(x) = c_0+c_1x+ c_2x^2$\\
Имеем следующие точки:
\[
(0;f(a)), (\frac{\omega}{2};f(\frac{a+b}{2})), (\omega;f(b))
\]
\[
\begin{cases}
c_0 = f(a)\\
c_0 + c_1\frac{\omega}{2} + c_2\frac{\omega^2}{4} = f(\frac{a+b}{2})\\
c_0 + c_1\omega + c_2\omega^2 = f(b)\\
\end{cases}
\overset{(2)\cdot4+(1)+(3)}{\to}
6c_0 + 3c_1\omega+2c_2\omega^2=f(a) +4f(\frac{a+b}{2}) + f(b)
\]
\[
\int\limits_{a}^{b} f(x) \, dx \approx  \int\limits_{a}^{b}(c_0+c_1x+c_2x^2)dx =c_0\omega+c_1\frac{\omega^2}{2} + c_2\frac{\omega^3}{3} = \frac{\omega}{6}(6c_0+3c_1\omega+2c_2\omega^2) =
\]
\[
=\frac{b-a}{6}(f(a)+4f(\frac{a+b}{2}) + f(b))
\]
Получили \textit{квадратурную формулу Симпсона}.\\
\[
\int\limits_{a}^{b} f(x) \, dx \approx\frac{b-a}{6}(f(a)+4f(\frac{a+b}{2}) + f(b))
\]
\textbf{Утверждение}\\
АСТ квадратурной формулы Симпсона = 3\\
\[
|R(f)| \leq \frac{M_4}{2880}(b-a)^3
\]
Как быстрее повышать степень?\\
\subsubsection{Квадратурные формулы Гаусса.}
Идея: Управлять не только весами, но и узлами.\\
Выберем узлы и веса так, чтобы максимизировать АСТ.\\
\[
\int_{a}^{b} f(x) \, dx  =  \sum\limits_{i=1}^{n} {c_k}  f(x_k)
\]
\textit{Нумерация с 1!}\\
Точность: $1,x,x^2,...,x^m$\\
\\
$
\begin{cases}
 \sum\limits_{i=1}^{n} {c_k}x^l_k = \frac{1}{l+1}(b^{l+1}-a^{l+1})\\
l=0,1,...,m
\end{cases}$\\
\\
Имеем 2n неизвестных и m+1 уравнений.\\
Тогда при m = 2n - 1 существует решение.\\
\\
\textbf{Утверждение}\\
Наивысшая АСТ квадратурной формулы построенной по n узлам не превосходит 2n - 1.\\
\\
Доказательство:\\
\[
g(x) = ((x-x_1)...(x-x_n))^2; \, deg(g(x)) = 2n
\]
\[
\int\limits_{a}^{b} g(x) \, dx > 0, \,g(x_k) = 0 \Rightarrow  \sum = 0
\,\,\,\,\,\,\,\,\,\,\,\,\,\blacksquare
\]\\
Как влияет выбор узлов на АСТ?
\[
\begin{array}{|c|c|c|}
\hline
\text{2 узла} & \text{Трапеции: 1} & \text{Гаусс: 3} \\
\hline
\text{3 узла} & \text{Симпсон: 3} & \text{Гаусс: 5}  \\
\hline
\end{array}
\]\\
\textbf{Утверждение}\\
Погрешность для квадратурной формулы Гаусса с 2 узлами:\\
\[
|R(f)| \leq \frac{M_4}{4320}(b-a)^5
\]\\
\textbf{Простейшие квадратурные формулы Гаусса}:\\
$\textbf{n=1} \Rightarrow m = 2n- 1= 1$\\
\[
\begin{cases}
c_1 = b -1\\
c_1x_1 = \frac{1}{2}(b^2-a^2)
\end{cases} \Rightarrow
\begin{cases}
c_1 = b -1\\
c_1x_1 = \frac{1}{2}(a+b)
\end{cases}
\]
\[
\int\limits_{a}^{b} f(x)\,dx = (b-a)f(\frac{a+b}{2})
\]
Получили формулу средних прямоугольников.\\
\\
$\textbf{n=2}\Rightarrow m = 2n- 1= 3$\\
\[
\begin{cases}
c_1 + c_2 = b - a\\
c_1x_1 + c_2x_2 = \frac{1}{2}(b^2-a^2)\\
c_1x^2_1 + c_2x^2_2 = \frac{1}{3}(b^3-a^3)\\
c_1x^3_1 + c_2x^3_2 = \frac{1}{4}(b^4-a^4)
\end{cases} \Rightarrow
\int\limits_{a}^{b} f(x)\,dx \approx \frac{b-a}{2}(f(x_1)+f(x_2));\,x_{1,2} = \frac{a+b}{2} \pm \frac{\sqrt{3}}{6}(b-a)
\]
Получили почти формулу трапеций, только в специальных узлах.\\

\textbf{Теорема}\\
Квадратурная формула Гаусса, построенная по n узлам, обладает наивысшей АСТ $\Leftrightarrow 2n-1 \Leftrightarrow\\$
1) Это интерполяционная квадратурная формула\\
2) Её узлы $x_1,..,x_k$ являются корнями многочлена:\\
$\omega(x) = (x-x_1)(x-x_2)...(x-x_k)$ такого, что:\\
$\forall q(x): \, deg(q(x)) \leq n-1 :\, \int\limits_{a}^{b}\omega(x)q(x) = 0$\\
\\
Откуда брать $q(x)$? Полиномы Лежандро (формула Родрига)\\
$L_n(x) = \frac{1}{2^nn!}\frac{d^n}{dx^n}(x^2-1)^n, \,x\in[-1;1]$\\
\\
Докажем ортогональность:\\
Пусть $\psi(x) = (x^2-1)^n$
\[
\psi^{(k)}(-1) = \psi^{(k)}(1) = 0 \Rightarrow L_n(x) = \frac{1}{2^nn!}\psi^{(n)}(x)
\]
$\exists Q(x) \in C^n[-1;1]$
\[
\int\limits_{-1}^{1} Q(x)L_n(x)\,dx = \frac{1}{2^nn!}\int\limits_{-1}^{1} Q(x)\psi^{(n)}(x)\,dx = \frac{1}{2^nn!}\int\limits_{-1}^{1} Q(x)\,d(\psi^{(n-1)}(x)) =
\]
\[
= \frac{1}{2^nn!}\left(Q(x)\psi^{(n-1)}(x)\bigg |^1_{-1}-\int\limits_{-1}^{1} Q'(x)\psi^{(n-1)}(x)\,dx\right) =
\]
\[
= -\frac{1}{2^nn!}\int\limits_{-1}^{1} Q'(x)\psi^{(n-1)}(x)\,dx = \, ...\, =(-1)^n\frac{1}{2^nn!}\int\limits_{-1}^{1} Q^{(n)}(x)\psi(x)\,dx
\]
Пусть $Q(x)$ --- полином, $deg(Q(x)) \leq n-1$\\
\[
Q^{(n)}(x) = 0 \Rightarrow \int\limits_{-1}^{1} Q^{(n)}(x)\psi(x)\,dx =0 \Rightarrow \int\limits_{-1}^{1} L_n(x)Q(x)\,dx = 0
\]
\textbf{Вывод}: Оптимальные узлы для квадратичной формулы Гаусса - корни полинома Лежандро.\\
$P_2(x) = \frac{1}{2}(3x^2-1)$\\
$P_3(x) = \frac{1}{8}(63x^5-70x^3+15x)$\\
$\dots$\\
$\underset{[a;b]}{X_k} = \frac{1}{2}(a+b) +\frac{1}{2}(b-a)\underset{[-1;1]}{X_k}$\\
$\underset{[a;b]}{C_k}=\frac{1}{2}(b-a)\underset{[-1;1]}{C_k}$\\
Извстно, что $C_k = \frac{2}{(1-x^2_k)(L'_n(x_k))^2}$ \textit{Символы Кристоффеля (1858))} \\
\\
Существуют также формулы Филона для быстро осциллирующих функций.
\subsubsection{Несобственные интегралы:}
\[
\int\limits_{a}^{b} f(x) \, dx = \frac{f(a)+f(b)}{2}(b-a) \approx \{[a;b] = \bigcup[x_i;x_{i+1}]\} \approx \sum\limits_{i=0}^{n-1}\frac{f(x_i)+f(x_{i+1})}{2}(x_{i+1}-x_i) \approx
\]\\
\[
\approx \{x_{i+1}-x_i=h -const\} \approx h(\frac{f_0+f_n}{2} + \sum\limits_{i=0}^{n-1}f(x_i))\approx h(c_0 + \sum\limits_{i=0}^{n-1}f(a+ih)) \Rightarrow
\]
\begin{equation}
\Rightarrow \int\limits_{-\infty}^{\infty} f(x) \, dx = h\sum\limits_{i=-\infty}^{\infty}f(ih)
\end{equation}
(22) --- формула трапеций  для бесконечного случая.\\
Теперь рассмотрим$\int\limits_{-1}^{1} f(x) \, dx $ с особенностями в -1 и 1.\\
\[
x = th(\frac{t}{2}) = \frac{e^{\frac{t}{2}}-e^{-\frac{t}{2}}}{e^{\frac{t}{2}} + e^{-\frac{t}{2}}} = \frac{e^t-1}{e^t+1}; \,dx = \frac{2e^t}{(e^t+1)^2}dt
\]
\[
\int\limits_{-1}^{1} f(x) \, dx = \int\limits_{-\infty}^{\infty} f(\frac{e^t-1}{e^t+1}) \, d\left(\frac{e^t-1}{e^t+1}\right) = 2\int\limits_{-\infty}^{\infty} f(\frac{e^t-1}{e^t+1}) \, \frac{e^t}{(e^t+1)^2}dt = 2h \sum\limits_{i=-\infty}^{\infty}f(\frac{e^{ih}-1}{e^{ih}+1})\frac{e^{ih}}{(e^{ih}+1)^2}
\]
Получили квадратурную формулу Стенжера.\\
\[
\int\limits_{-1}^{1} f(x) \, dx  \approx \sum\limits_{i=-N}^{N}f(\frac{e^{i}-1}{e^{i}+1})\frac{e^{i}}{(e^{i}+1)^2}
\]
Известно, что ошибка пропорциональна $O(e^{-\pi\sqrt{N}})$
\subsubsection{Кратные интегралы:}
\textbf{1)}
$
G = \begin{cases}
a\leq x\leq b\\
c\leq y \leq d
\end{cases}
$
\[
\iint\limits_{D} f(x,y) \, dx \, dy = \int\limits_{a}^{b}dx\int\limits_{c}^{d}f(x,y)\,dy \overset{Симпсон}{=} \frac{b-a}{6}\left(\int\limits_{c}^{d}f(a,y)\,dy + 4\int\limits_{c}^{d}f(\frac{a+b}{2},y)\,dy + \int\limits_{c}^{d}f(b,y)\,dy\right)
\]
Отдельно применяем формулу Симпсона для каждого интеграла.\\
\\
\textbf{2)}
$G \subset 	\mathbb{R}^2$\\
Можем приближать исходную область прямоугольниками.\\
Можем делать интерполяцию f(x,y).\\
\\
\textbf{3)}
$G \subset 	\mathbb{R}^3$\\
Метод Люстерника и Фиткина.\\
М.П. Мысовских "Интерполяционные кубаторные формулы".\\
Метод Монте-Карло.
